\documentclass[french]{article}
\usepackage[T1]{fontenc}
\usepackage[utf8]{inputenc}
\usepackage{lmodern}
\usepackage[a4paper]{geometry}
\usepackage{babel}
\usepackage{amsmath}
\usepackage{amsfonts}
\usepackage{mathtools}
\usepackage{tcolorbox}
\usepackage{color}
\usepackage{breqn}

\begin{document}
	\title{L'Ensemble Microcanonique}
	\author{}
	\date{}
	
	\maketitle
	
	\section{Défauts dans un cristal: ensemble microcanonique}
	\begin{tcolorbox}[colback=blue!5!white,colframe=blue!75!black]
		\quad Dans un cristal parfait, les atomes occupent les sites d’un réseau périodique (cubique par exemple). Certains atomes peuvent sortir de leur position d’équilibre pour occuper des sites dits interstitiels (situés par exemple entre deux atomes) mais avec une pénalité en énergie $+dE$.
		
		\quad Un cristal contient $10^{24}$ atomes pouvant occuper des sites interstitiels, et $dE=0.1eV$. On crée une population de sites interstitiels d’énergie totale $4000J$. Ce système est isolé de l’extérieur.
		
		\quad On observe l’état de l’un des atomes. Quelle est la probabilité de l’observer dans sa position interstitielle?
	\end{tcolorbox}

	\section{Fluctuations de densité dans un gaz parfait}
	\begin{tcolorbox}[colback=blue!5!white,colframe=blue!75!black]
		\quad Une bouteille d’$1$ litre contient une mole de gaz, et elle est bien isolée de l’extérieur. Ce gaz est supposé parfait, ce qui signifie que, si l’on néglige la force de pesanteur, l’énergie est indépendante des positions des molécules du gaz.
		
		\quad On regarde à un instant donné un petit volume de la bouteille qui est un cube de $0.01$ micron de côté. Quelle est la probabilité qu’il n’y ait aucune molécule dans ce volume ?
	\end{tcolorbox}


	\section{Information génétique et cerveau}
	\begin{tcolorbox}[colback=blue!5!white,colframe=blue!75!black]
		\quad L'ADN humain contient environ $3$ milliards de paires de bases. Le cerveau humain contient environ $10^{11}$ neurones, chacun étant connecté à environ 	$410^3$ synapses.
		
		\quad On suppose pour simplifier que les synapses peuvent être seulement de deux types, excitatrices ou inhibitrices. Y a-t-il assez d'information dans l'ADN pour coder l'état de chaque synapse?
	\end{tcolorbox}
	
	
	
\end{document}