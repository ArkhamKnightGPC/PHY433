\documentclass[english]{article}
\usepackage[T1]{fontenc}
\usepackage[utf8]{inputenc}
\usepackage{lmodern}
\usepackage[a4paper]{geometry}
\usepackage{babel}
\usepackage{amsmath}
\usepackage{amsfonts}
\usepackage{mathtools}
\usepackage{tcolorbox}
\usepackage{color}
\usepackage{breqn}

\begin{document}
	\title{The Microcanonical Ensemble}
	\author{}
	\date{}
	
	\maketitle
	
	\section{Defects in a crystal: microcanonical ensemble}
	\begin{tcolorbox}[colback=blue!5!white,colframe=blue!75!black]
		\quad In a perfect crystal, atoms occupy the sites of a periodic lattice (cubic, for example). Some atoms may leave their equilibrium position to occupy so-called interstitial sites (located, for example, between two atoms) but with an energy penalty of $+dE$.
		
		\quad A crystal contains $10^{24}$ atoms capable of occupying interstitial sites, and $dE=0.1$ eV. We create a population of interstitial sites with a total energy of $4000$ J. This system is isolated from the outside.
		
		\quad We observe the state of one of the atoms. What is the probability of observing it in its interstitial position?
	\end{tcolorbox}

	\section{Density fluctuations in an ideal gas}
	\begin{tcolorbox}[colback=blue!5!white,colframe=blue!75!black]
		\quad A $1$ liter bottle contains one mole of gas, and it is well isolated from the outside. This gas is assumed to be ideal, which means that, if we neglect the force of gravity, the energy is independent of the positions of the gas molecules.
		
		\quad At a given instant, we look at a small volume of the bottle which is a cube of side $0.01$ micron. What is the probability that there are no molecules in this volume?
	\end{tcolorbox}


	\section{Genetic information and the brain}
	\begin{tcolorbox}[colback=blue!5!white,colframe=blue!75!black]
		\quad The human DNA contains approximately $3$ billion base pairs. The human brain contains approximately $10^{11}$ neurons, each being connected to approximately $410^3$ synapses.
		
		\quad We assume, for simplification, that synapses can only be of two types, excitatory or inhibitory. Is there enough information in the DNA to encode the state of each synapse?
	\end{tcolorbox}
	
	
	
\end{document}
